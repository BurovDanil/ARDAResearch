\documentclass[]{report}
\renewcommand\thesection{\arabic{section}}%for page numbering in arabics
\usepackage{graphicx,tabularx}%for figures and tables
\usepackage[utf8]{inputenc} %allows special characters such as ä, ö, ỳ
\usepackage[english]{babel}  %set the language to English
\usepackage[margin=1.5in]{geometry} %change page margins 
\usepackage{sectsty}%section headers
\allsectionsfont{\sffamily\large}
\subsectionfont{\sffamily\normalsize}
\linespread{1.2}% line distance
\usepackage{lipsum}% http://ctan.org/pkg/lipsum
\usepackage{caption}%use for captions on tables
%use this exact command. The style and bibliographystyle has to be authoryear (Havard). The sorting is nyt: name, year, title so that the bibliography is sorted alphabetically. firstinits=true shortens the names: Albert Einstein -> A. Einstein
\usepackage[backend=bibtex,style=authoryear,bibstyle=authoryear,sorting=nyt,firstinits=true]{biblatex}
\setlength\parindent{0pt}%include this so that your paragraphs don't indent automatically
\addbibresource{report.bib} %this attaches your bib-file, your bibliography (must be in the same folder)
\usepackage[compact]{titlesec}%include title formatting package

% Title Page
\title{Git CLI or GitHub Desktop \\ Which interface is more efficient and faster?}
\author{Danil Burov and Giulio Raffaeli}
\date{December 18th 2023\\Module: ARDA \\Venlo, Limburg, Netherlands}


\begin{document}
	
	\maketitle
	
	\begin{abstract}
		%history%
		From the 1970s, Command Line System (CLI) enables users to input commands for executing computers tasks. As technology progressed, graphical user interfaces (GUI) like Xerox Alto (Xerox - 1973) and the first Macintosh (Apple - 1984) became the mainstream choice. GUIs are the primary interface for most users because of the user-friendly experience, relegating CLI only to experienced users.\\
		%cite -> wikipedia% 
		%topic introdutction%
		This research is an attempt to evaluate whether using Git with CLI is more efficient and faster compared to its counterpart, GitHub Desktop. In order to analyze the  differences between the effectiveness of using one or the other, a survey was conducted among IT student. Along with a short experiment.\\		
		%data evaluation%
		The results show that using the CLI instead of using the GUI for executing basic Git commands is faster and more efficient.   After getting to know the CLI commands the average pace of executing them increased significantly, showing how using the CLI can be more effective overtime. Moreover,some people who had never used the CLI before, showed interest in using it for personal usage.
		\pagenumbering{roman}
		
	\end{abstract}
	\tableofcontents
	\setcounter{page}{3}
	\listoffigures %UNCOMMENT IF YOU HAVE FIGURES
	%\listoftables %UNCOMMENT IF YOU HAVE TABLES
	\pagebreak
	
	\pagenumbering{arabic}	
	
	\section{Introduction}
	The purpose of this chapter is to provide the research question and the importance of it, along with context about the topic. Furthermore, this chapter will point out potential external factors that may influence the evaluation of the hypothesis. \\
	%(Context and background)git history%
	%CLI and GUI%
	%Reasearch question% 
	%hypothesis -> X -> Y potential C's and I's%
	\subsection{Research questions and Hypothesis}
	The aim of the research is to prove that using the CLI for Git related tasks is more superior when comes down to efficiency and speed compared to the GitHub Desktop . Therefore our research question is : "Which interface is more efficient and faster between them?".\\
	
	%%here we stopped---------
	We hypothesis that using CLI instead of GitHub Desktop has much more benefits in terms of 
	%result evaluation%
	This is the introduction.
	\subsection{Some subsection}
	
	This is the first subsection of the introduction.
	\section{Methodology}
	%Research model -> research methods%
	%Survery%
	%Experiment%
	\lipsum
	\section{Results}
	%Survery results%
	%Experiment results%
	%Data analysis%
	\newpage	
	\section{Discussion}
	%Result interpration%
	%Evaluate research questions%
	%Limitations -> Y factors%
	%further research%
\end{document}


\printbibliography[title=References]

\end{document}          
